\documentclass[letterpaper]{article}
\setlength{\columnsep}{0.75cm}
\usepackage{algorithm2e}
\usepackage{multirow}
\usepackage{graphicx}
\usepackage{amsmath}
\usepackage{cite}
\usepackage{framed, color}
\usepackage{soul}
\definecolor{shadecolor}{rgb}{.93,.93,.93}
\definecolor{blu}{rgb}{0,0,1}
\newcommand{\td}[1]{{\color{blu}\hl{TODO: #1}}}
\definecolor{shadecolor}{rgb}{.93,.93,.93}
\usepackage{amssymb}
\usepackage{mathrsfs}
\usepackage{gensymb}
\usepackage[hmargin=1in,vmargin=1in]{geometry}
\usepackage{scrextend}
\setlength{\parindent}{0cm}
\setlength{\parskip}{4mm plus1mm minus1mm}
\usepackage[colorlinks=true,urlcolor=blue]{hyperref}

\title{}
\author{}
 
\parindent0pt \parskip8pt
\begin{document}
\maketitle
Background \\
- Text classification has an enormous number of practical applications, from
	spam filtering\cite{wang2013}, to triaging helpdesk requests, to 
	recommender systems (using review text)\cite{jannach2010recommender}.  An 
	exciting recent application 
	is the inference of latent attributes about a user based on their posts in 
	social media \cite{liu2013s,liu2012using,al2012homophily}.

- There is is a rich body of literature that has explored many aspects of
	the text classification problem.  Chosing the set of features to represent
	documents is critical to accurate classification, and various options 
	have been investigated\cite{leopold2002text, wang2013}.  Features may
	consist of frequency counds, boolean variables that only indicate whether
	a word is present (but don't show its frequency)\cite{leopold2002text}.  
	But often, before feautures are calculated, some pre-processing of the
	documents is often done.  The most 
	frequent and most rare words are sometimes ignored because they tend to
	be less descriminating \cite{leopold2002text}.  
	A list of stopwords can be used, which is a set 
	of common words deemed to be not useful for text classification purposes 
	\cite{leopold2002text}.
	The words in the documents may also be simplified by stemming or 
	lemmatization, for example, to convert \texttt{running} and \texttt{ran}
	both to the more basic infinitive form \texttt{run} \cite{leopold2002text}.
	Once Feature vectors are obtained, they can be 
	further treated by multiplying the feature vectors (component wise) by a 
	weighting scheme\cite{leopold2002text}.
	The application of weights has a dual purpose.  
	Primarily, it regulates the degree of influence that given features have
	over the decision boundary to be learned, which ideally should be made to
	reflect the degree to which given features are descriminative 
	\cite{leopold2002text}.  
	Secondarily, it can
	be used to normalize the vectors to unit length according to some metric
	(usually the euclidean distance or L2-norm)\cite{leopold2002text}.  
	Normalization helps 
	overcome the fact that longer documents have higher frequencies overall, 
	which otherwise can be a source of bias\cite{leopold2002text}.

	In addition to choosing how to represent the data, one must choose the
	learning algorithm.  The performance of learning algorithms have been
	scrutinized in a wide variety of contexts.  Roughly ordered in order
	of increasing accuracy, the algorithms that have been found useful in
	text classification include: naive Bayes (NB), k-nearest-neighbor (kNN), 
	latent Dirichlet allocation (LDA), and support vector machines (SVMs).
	



Choice of inference method (learner) \\

Preprocessing \\
- lematization offers little or no benefit when using SVM 
	\cite{leopold2002text}\\

Normalization \\
- l2-normilization is helpful for SVM \cite{leopold2002text}

Choice of term weighting scheme \\ 
- The choice of weighting scheme is more important than the choice
	of kernel \cite{leopold2002text}\\
- redundancy does better than idf\cite{leopold2002text} \\

SVM
- the rbf kernal appear beats linear and polynomial \cite{leopold2002text} \\

Performrance:\\
Naive bayes:\\
	- lematized: 0.477466

\bibliography{bib}
\bibliographystyle{plain}
\end{document}


