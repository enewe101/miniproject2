\documentclass[conference,letterpaper]{IEEEtran}

\IEEEoverridecommandlockouts
\overrideIEEEmargins

\usepackage{cite} 
\usepackage{graphicx}  
\usepackage{amsmath}   
\usepackage{multirow}
\usepackage[left=0.71in,top=0.94in,right=0.71in,bottom=1.18in]{geometry}
\setlength{\columnsep}{0.24in}

% correct bad hyphenation here
%\hyphenation{op-tical net-works semi-conduc-tor IEEEtran}

\title{}
\begin{document}
% paper title
\title{\huge 
	COMP-598 Miniproject2: Abstract classification\\
	team \textit{Renegade Masters}
}

% author names and affiliations
\author{\authorblockN{Ibrahim Abdelghany, Hardik Vala, and Edward Newell}
\authorblockA{
	\textit{School of Computer Science}; \textit{McGill University}\\
\textit{Montreal, Quebec, H3A 0G4, Canada}\\
\textit{\{ibrahim.abdelghany, hardik.vala, edward.newell\}@mail.mcgill.ca}\\}%
}
% make the title area
\maketitle
\begin{abstract}
These instructions give you the basic guidelines for preparing
papers for icma2012/IEEE conference proceedings.To get more information,
please visit our website: http://www.ieee-ICMA.org
\\
\end{abstract}

% key words
\begin{keywords}
List key index terms here. No more than 5.
\end{keywords}

\section{Introduction}
The quantity of data available contitinues to grow, as does it's rate of 
production, with no sign of letting up.  Most of the worlds information is
in text form, which is both easily read by humans and, when used to relay
an idea, is often more compact than other methods, such as images and 
sound.

Text classification has its origins in information retrieval (IR), as do 
many of the methods we use here.  IR differs from text classification in the
kind of problem it is trying to solve.  In IR, documents are not seen as 
belonging to classes.  Instead, the task is to determine, for a given 
query, which are the most relevant documents.  Because the query is not
fixed, one connot think of documents as belonging to a relevant and 
irrelevant set (at least until some query is given), so any pre-processing
or representation methods that attempt to improve the retrieval of relevant 
documents must be carried out in an unsupervised fashion.

In the early 2000's, two methods emerged as front-runners in text 
classification task.  These methods were not mutually exclusive, one being
an approach to representation, that of LDA, and the other being an approach
to classifier learning, SVM.  SVM was found to be 
highly performant, beating k-Nearest Neighbour and multinomial Naive Bayes
algorithms.  In optimizing representation fed into SVM, it was found that
applying a relevancy weighting to word counts improved the performance of
the resulting classifier substancially---generally having a more important
effect than the choice of kernal or tunable parameters.  Two representations
emereged that improved SVM accuracy.  One was TFIDF, which was borowed 
directly from IR, and another was TFICF, which is based on a similar notion
to TFIDF, but takes into account the example classes.  These respectively 
improved SVM.  But the application of SVM to LDA creates in many cases,
an even more performant approach to induction.  Both methods are space
efficient, and once the LDA features are extracted, SVM is time-efficient
during learning.
learning 

Here we review a broad set of representations and classifiers in a text
classification task.  We use a set of abstracts for 
\section{Related work}
- stuff about information retrieval
- stuff about text classification in general
- stuff about abstract classification specifically
- stuff about the progress in algorithms


The LDA feature representation was introduced in 2003, building on the
ideas of LSI.  The purpose of the LDA representation is to identify a 
much lower dimentional subspace of the word-frequency feature space that still 
contains most of the variation between documents.  This lower-dimensional
manifold has `topics' as its basis vertors, where each topic represents a 
probability distribution over words.  Documents are seen as being made of
mixtures of topics, and each word is seen as having arisen as the random
draw from one of the topic word distributions.

Not only does the reduction in dimentsionality make subsequent inference
algorithms more efficient, it actually improves the results in many cases.
In much the same way that TFIDF identifies which words are relevant to the
classification task, LDA forces the classifier to focus a small subset of
features which, because they are capable of expressing a large amount of the 
variation, are likely to be relevant for classification.


\section{Data pre-processing methods}
The raw form of the data consisted of the full text of abstracts and the
class of the abstract, which was either `cs', `stat', `math', or `physics'.
The dataset contained some problematic entries.  For example, sometimes we
found that the abstract was actually only a statement that the paper had been
removed.  In other cases, we found what appeared to column headings (e.g.
``abstract'' and ``class'') within data rows.  We removed all of these entries
manually.

Before extracting features from the text, we did various things to improve
results.  First, all the text was lowercased, and all punctuation was 
removed.  However, this step would tend to alter latex commands, numbers,
and urls.  We felt that the mere occurrence of these types of elements would
be potentially descriminitive, so we included an optional pre-processing step
that matched such types using regular expressions, and replaced them 
respectively with the tokens `aNUMBER', `aLATEX', `aURL', so that counts of
these types could be extracted later.  We will refer to these types as 
``specials''.

Coding of the ``specials'' constitutes a dimensionality reduction.  We also
performed other dimensionality reductions, including the removal of 
stop words and lemmatization.  To do both of these, we used the
wordnet corpus of the nltk package.

 
\section{Feature design and selection}
We tried several alternative feature representations.  The simplest of these
was the binary representation, which simply recorded, for each possible 
word in the abstracts, which words were actually present.  As a technical
point, encoding this as a feature vector can take up a very large amount of
space, because it would involve indicating a `0' for all words that 
\textit{did not occur} in a given abstract.  In general, for all our 
representations, we used an key-based encoding, such that the type (word)
is used as a key, and the associated value, while tokens whose value is 
zero are simply omited.  Another very similar feature set we used involved 
indicating the actual counts used.

We derived two refinements of the representation using word-counts, by applying
a multiplicative weight to the counts.  It has been found that, at least for
one of the most successful induction algorithms---support vector machines 
(SVM)---applying multiplicative factors to frequency counts is far more 
important to the performance than the selection of the kernel and tuning of 
parameters within the algorithm.  Therefore, we also prepared several 
representations in which the standard frequency count features were augmented
by applying a multiplicative factor.

The first, TFIDF (term frequency inverse document frequency) is based 
on a relevance metric from information retrieval.  It provides a greater 
weight to those words which occur in 
few documents, based on the notion that these words are more descriminative.
TFIDF weights were calculated using the following formula:

\begin{align}
	\textsc{tfidf}(t_i, d_j) = \textsc{tf}(t_i, d_j) \cdot \textsc{idf}(t_i) \\
	= \textsc{tf}(t_i, d_j) \cdot \lg \frac{N}{n(t_i)}
\end{align}

However, because TFIDF was developped for the information retrieval application
it does not take into account the extent to which a token appears in 
each class.  One might expect that a token which is quite common in one class
but does not appear at all in another class, might be highly descriminative
for the classification task; meanwhile a token which is quite rare overall, 
but equally represented in all classes, might not aid in the classification
task.  Thus tficf was introduced by X.  The results of X showed that 
this weighting provided a performant representation, beating several 
information-theoretic weighting schemes.  The TFICF weight for term $t_i$ and
document $d_j$, which is denoted $\textsc{tfidf}(t_i, d_j)$, is calculated 
as follows:
\begin{align}
	\textsc{tfidf}(t_i, d_j) = \textsc{tf}(t_i, d_j) \cdot \textsc{icf}(t_i) \\
	= \textsc{icf}(t_i, d_j) \cdot \lg \left(1 + \frac{C}{cf(t_i)}\right),
\end{align}
where $\textsc{tf}(t_i, d_j)$ is the frequency of term $t_i$ in document $d_j$,
$\textsc{icf}(t_i)$ is the inverse \textit{class} frequency, $C$ is the 
total number of classes, and $cf(t_i)$ is the number of classes containing at
least one document with an occurence of $t_i$.


Continuing the reasoning by which TFICF was developed, one might go further.
Suppose one has a token which is very frequent in one class, but which still
occurs only a few times in each of the other classes.  One would expect this
to be almost as descriminative as a feature which occured frequently in one
class but not at all in another.  Based on this reasoning, we developped
\textit{modified TFICF}, using the following formula:
\begin{align}
	\textsc{mtfidf}(t_i, d_j) 
		&= \textsc{tf}(t_i, d_j) \cdot \textsc{micf}(t_i) \\
	\textsc{micf}(t_i)
		&\equiv \lg \left(1 
			+ C\sum_j^C \frac{1}{j}\Big(n_{j}(t_i) - n_{j+1}(t_i)\Big) \right)
\end{align}

Our final representation used an unsupervised technique to detect the
underlying `topic' composition of abstracts, using latent dirichlet allocation.
This technique works on the vector representation of words, and looks for
a set of vectors which account for most of the variation seen between examples,
in an unsupervised way.  Each of these vectors, or `topics' represents the
presence of a mixture of word-frequencies that tend to co-occur (when their
components are all positive) or be absent.  The algorithm therefore reduces
the dimensionality of the space of abstracts dramatically, by `explaining'
variations in frequency counts as variation in a much smaller number of 
topics.  The topics constitute a new basis for a subspace of the original
space of abstracts.

We considered applying normalization techniques to our features, but chose
not to because we already had a very large number of possible pre-processing
and feature representation combinations to explore, and because the abstracts
tended to be all of a similar length, so we reasoned that normalizing vertor 
lengths would probably not alter the performance of the induction algorithms 
very much.

\section{Algorithm selection}
\paragraph{Baseline algorithm}
We chose a Naive Bayes classifier as our baseline method, implemented to 
represent the set of word-frequencies of abstracts as a class-conditional 
multinomial distribution.  This algorithm operates on the training data
by keeping a tally of the frequency with which each token appears in a given
class, and using the frequency of appearance to directly estimate the 
token probability.

Technically, multinomial distributions are supported over integer-valued
frequency vectors (because a token cannot appear, e.g. 1.5 times in a 
document).  However, there is no practical barrier to using the weighted
representations, such as TFIDF and TFICF.

In terms of implementation, we found that validation and testing of the Naive 
Bayes algorithm could be made highly performant by implementing it in a 
lazy way: the algorithm keeps a tally of frequencies, but only calculates 
the necessary conditional probabilities for features when called upon to
classify a given new example.  Because the raw frequencies are stored,
the classifier can be re-trained on a different subset of the training data
by simply adding or subtracting from the tallies based on which examples are
newly present (or absent) in the new subset.  This makes cross validation much
faster.  As a result, we were able to systematically investigate all 
combinations of pre-treatment and feature representation, including LDA.
For the LDA representation, we also used an implementation of Naive Bayes 
which assumes that word frequencies have a Gaussian distribution.  Again
this distribution embeds an impossible assumption that words can occur a 
fractional number of times, but there is in fact no barrier to using the 
such an implementation in practice.

\paragraph{Standard algorithm}
For our standard algorithm, we implemented k-nearest neighbor classification.
This method works by assuming a distance metric between examples within the
space defined by word frequencies.  The logic of using such an algorithm is 
that words which are close, and hence have similar word-frequencies, are more
likely to be from the same class.  An important optimization in this 
algorithm is the selection of a distance metric.  While the euclidean distance
metric is the most intuitive, it is not necessarily the most suitable for
a given domain.  In text classification, prior results have shown that 
cosine similarity (a pseudo-distance metric because it does not obeay the
triangel inequality) is a better choice.  We investigated both techniques.

Unlike the Naive Bayes classifier, kNN becomes computationally very intensive
as one increases the number of examples.  This is because, for each query
example, the algorithm must search through the entire training data set to
find the k nearest neighbors.  This complexity increases linearly with the
size of the training set data, the dimensionality of the data, and the 
number of neighbors $k$.  As a result, we were forced to tradeoff these values
for the algorithm to be feasible given our computational resources.  There
exist known data structures that reduce the number of distance calculations and
comparisons that need to be made within the kNN algorithm, however, based on
our early results from this algorithm, which were not very promising, we 
decided to spend more time optimizing the most performant algorithms rather
than pursuing an computationally efficient kNN algorithm.


\paragraph{Advanced algorithms}
We chose to investigate several advanced algorithms, given the existance of
high quality and easy-to-use packages available for text classification.
We used the mallet Java package to produce the LDA feature representation, 
and then used a variety of classifiers based on the LDA features.

Using the LDA representation, we performed the multinomial Naive 
Bayes, Gaussian Naive Bayes, logistic regression, random forest, decision tree,
and SVM algorithms.  In this case, because of the wide variety of 
implementations we decided to try as many as possible.  We will come back to
this point when we discuss the ensemble classifier.  

We also tried SVM on the feature counts, weighted by TFICF.  We chose this 
weighting outright according to the results of X, who showed that this was
the most efficient representation, having also considered X,Y, and Z.  and 
weighted feature representations.  For variety, we chose Scikit-learn's 
implementation, rather than re-using Mallet's (which we used on LDA topics).

The major reason for focussing on testing a variety of classifiers is that
we will combine the various configurations of classifiers using an ensemble
classifier.  We will do this by producing predictions from all of the 
classifiers individually, and treat these as features.  Since each 
feature learns a different `concept' for the classes, there biases will
tend to be different.  On top of these features we use a decision tree.
The decision tree will rely on the output of different classifiers, based 
on their track-records during cross-validation.

To build the training set for the decision tree-based ensemble classifier,
we record the predictions that each classifier makes during cross-validation.
Thus, we get predictions by thos classifiers for all of the training data,
but the prediction always made about an example that the classifier has 
not seen.  After cross validation, we trained a each classifier on the
entire training set, using the optimal settings, and then classified the test
data.  This produces our base-classifier feature set.  We then  
trained the decision tree using the entire set of predictions.

At this point the reader may have expected us to cross-validate the 
ensemble classifier.  Theoretically we could do this, but the computational
complexity is prohibitive.  In fact, we had to make some trade-offs in 
constructing the base classifiers to negotiate their complexity.  Since we
intended to submit the ensemble classifier, we chose not to leave aside any
data in our training set since this would sacrifice our accuracy in the
competition.  In our case, retraining the decision tree-based ensemble
classifier, after an initial validation on partial data, requires
retraining all of the base-classifiers as well.  Instead, 
\textit{and only for the ensemble classifier}, we relied or our leaderboard 
score as an estimate of our true error.

\section{Optimization}
\begin{figure*}
	\centering
	\includegraphics[scale=0.45]{figs/NB_representations.pdf}
	\caption{Caption\dots}
	\label{fig:NB_representations}
\end{figure*}
Our optimization spanned multiple levels.  First, by trying as many kinds
of representations and classification algorithms, we attempted to optimize 
the representation and inductive methods.  In some cases we were able to try
all the representations with a given classifier, such as with Naive Bayes for 
example.  Additionally, we optimized parameter selections for specific
induction methods, which we describe in the next section.

\paragraph*{Optimizing preprocessing for Naive Bayes}
Since the \textsc{NB} algorithm was not 
computationally intensive, we used it to systematically investigate all of the
various options for feature representations.  For each of the basic 
representations (\textsc{tf}, \textsc{tfidf}, \textsc{tficf}, and 
\textsc{mtficf}), we tried all combinations of the pre-processing options, 
that is, with and without lemmatization, stop word removal, finding ``special 
tokens'', and using digrams.

The performance of the most basic representation, the term frequency 
(\textsc{tf}), provides a good entry point to understanding the effects
of all of these options (see fig X).  First observe that, when ranked by 
performance, the \textsc{tf}-based representations occupy a wide range of 
ranks.  However, the \textsc{tf} representations are always clustered together
into groups of four.  Looking within these clusters, it is apparent that
that including digrams, removing stop words,
or doing both, successively improve the result of \textsc{tf} approaches,
while lemmatization, and finding ``special tokens'' always had a slightly
adverse effect.

The \textsc{tfidf} representation was, in general, less affected by the
pre-processing options, which can be seen (in fig X) because all the
\textsc{tfidf} approaches are clustered together within an upper middle 
range of ranks.  So, \textsc{tfidf} seems to make the classifier more robust
against making the wrong choices.  However when the optimal pre-procesning 
choices are made for each basic representation, \textsc{tfidf} performs
the worst.  Interestingly, lemmatization and finding special tokens helped
\textsc{tfidf}, unlike for all of the other representations.  
Overall, \textsc{tfidf} moderates the performance, and improves it
overall, but the best performance was not achievable with it.

The \textsc{tficf} representation performed very well.  For a given set of 
options, it always performed better than \textsc{if}, and performed best 
overall when the optimal options were chosen.  It responded to options in the 
same
way as \textsc{if}, namely, including digrams, and especially removing stop
words was helpful, but lemmatization and finding special tokens was not.

The last basic representation was our experimental \textsc{mtficf}.  
Unfortunately it tended to perform slightly worse than plain \textsc{tf} for
a given selection of options.  However, for the optimal selection of options,
it did outperform \textsc{tfidf}.  Admittedly, scoring the 
features based on their contributions to the $l_1$-distance between the 
class-conditioned distributions is very similar to what the Naive Bayes 
algorithm already accomplishes by estimating the conditional probabilities.
Therefore, using \textsc{mtficf} in connection with NB might merely 
over-emphasizes the basis for descrimination that is already used.


\paragraph*{kNN distance metric}
We next describe optimizations for the kNN algoritm.  One of the most important
choices to be made in implementing a nearest-neighbour approach is to decide
on to measure distance.  The Euclidean distance is in many ways
the most intuitive, however, it is difficult to justify why it would be
meaningful in the case of text classification.  One reason to doubt its 
usefulness is that, for very high dimensional spaces (such as we have), 
the euclidean distance suffers from ``sphere hardening'', in which all points
begin to seem far from one another.

Cosine similarity provides an alternative with a good track-record as a 
similarity measure for text documents.  Technically cosine similarity is not
a true distance metric, because it does not satisfy the triangle inequality,
but this leads to no technical problems in using it in the kNN algorithm.
Intuitively, because cosine similarity looks only at the angle between 
the feature representations of documents, it is not affected by sphere 
hardening.  

For reference, the formal definitions for the euclidian distance, $L_2$ and
cosine distance, $L_\mathrm{cos}$, calculated between two documents having 
feature vectors $\mathbf{x}_1$ and $\mathbf{x}_2$ are as follows:
\begin{align}
	L_2(\mathbf{x}_1, \mathbf{x}_2) 
		&= \sqrt{\sum_{m=1}^M (x_2^{(m)} - x_1^{(m)})^2}, \\
	L_\mathrm{cos}(\mathbf{x}_1, \mathbf{x}_2) 
		&= 1 - \frac{\mathbf{x}_1 \cdot \mathbf{x}_2}
			{||\mathbf{x}_1|| ||\mathbf{x}_2||},
\end{align}
where $x_1^{(m)}$ designates the $m$th component of the vector $\mathbf{x}_1$,
and where `$\cdot$' signifies the dot-product (inner product) of two vectors,
and $||x_1||$ signifies the $L_2$-norm, which is the euclidean distance from
the of the vector relative to the zero vector: $L_2(\mathbf{x}, \mathbf{0})$.

Substituting the cosine distance in place of the Euclidean distance when 
training on 10000 examples dramatically increased accuracy, which reached 0.71,
up from 0.482.  These tests used feature reduction based on \textsc{tficf} 
scores, which we explain next.

\paragraph*{kNN feature reduction}
The kNN algorithm tends to be more susceptible to irrelevant
features than the NB and LDA algorithms, which limit the impact of 
non-descriminative features.
This susceptibility comes from the fact that kNN necessarily attributes equal 
importance to all features, up to scaling of the feature axes 
(and assuming isotropic distance metrics are used).  Thus, eliminating the
least descriminative features outright can improve accuracy with kNN.

Another reason to limit the number of features was that the kNN algorithm was
very computationally intensive.  Thus, our feature reduction decisions reflect
both an attempt to improve accuracy as well as to make executing the algorithm
on many training examples feasible.
To limit the number of features in a principled way, we employed lematization, 
stop-word removal, and binning special tokens.  We also chose not to include
digrams, even though it might improve the accuracy, because it would also more 
than double the number of features.

In addition we tried rejecting any terms that had $\textsc{tficf} \leq 1$.  
Applying a cutoff for \textsc{tfidf} is a common
way of diagnosing words that are universally too prevalent, and is a way to
produce a stop-word list.  However, based on our results from the NB algorithm,
it appeared that \textsc{tficf} would be a good choice for this.  When
trained on 5000 examples using the cosine distance metric, eliminating 
the features for which $\textsc{tficf}\leq 1$ improved the accuracy from 0.65
to 0.69.  This suggests that \textsc{tficf} is a good metric for determining
which terms have descriminative value in the classification task.

\paragraph*{kNN optimizing number of neighbours}
\begin{figure}
	\centering
	\includegraphics[scale=0.7]{figs/k_opt.pdf}
	\caption{Caption\dots}
	\label{fig:k_opt}
\end{figure}
In the k-Nearest Neighbours algorithm, we were fundamentally limited by 
the computatinal intensity of the algorithm.  We are aware of solutions for
this challenge, such as the use of $k$-d trees, which reduce the burden of
finding the nearest neighbours.  However our initial findings were not 
promising, and so we made a strategic decision to focus a greater amount of
effort on optimizing the most promising approaches.  Nevertheless, by limiting
the algorithm to work on a smaller subset of the training dataset, we 
investigated the effects of changing the neighborhood size $k$, and of using
alternative distance metrics.

To investigate the effects of $k$, we randomly sampled 250 examples from the
training data, and estimated the classification accuracy using 
cross-validation (our approach to validation is described in a dedicated 
section below).  

The results show a distinct peak for a neighborhood size of 6 to 8 examples.
Based on how kNN works, it is expected that neighborhood sizes that are 
too small yield predictions with very high variance, whereas neighborhood 
sizes that are too low lead to predictions that reflect the class priors
rather than being dependant on the examples that are very close to the query 
example.  Our 
observations are in-line with these tendencies.  We do remark, however, that
including all of the training data would likely shift peak accuracy to the
right (i.e. giving optimality for larger neighborhoods), since the density
of examples in all areas would increase, and so larger values of $k$ would 
still represent neighborhoods that are quite close to the query point.

\paragraph*{LDA: optimizing number of features}

\begin{figure}
	\centering
	\includegraphics[scale=0.7]{figs/LDA_num_features.pdf}
	\caption{Caption\dots}
	\label{fig:LDA_num_features}
\end{figure}

LDA is an unsupervised algorithm, however it is necessary to specify the 
number of features, or topics, that the algorithm should find.  Topics
should not be confused with classes.  Topics represent, in a geometric sense,
the vectors of greatest variation, in the space of word counts. Taken together,
these topics, span a 
subspace of the original feature space; this subspace is chosen in such a way
that when the documents are projected onto it, most of the variation between
documents is preserved.  The vectors are called ``topics'' because, often, 
the vectors major components (terms) that are all related to a particular 
identifiable topic.  By identifying a restricted set of topics that explain
most of the variation of the documents, LDA produces a set of features that 
are likely to be more informative to downstream classifier algorithms than
the original word frequencies.

To use LDA, it is necessary to specify the number of features (topics) that
are to be extracted.  We repeatedly ran LDA, while specifying different numbers
of features, to see the effect on a suite of downstream classifiers 
(see fig X).  
We investigated using between 4 and 256 topics; beyond 256 features the 
the feature extraction became too computationally intensive.  

In general, the dowstream classifiers responded in one of two ways.  Either
the classifier continually improved as more features were added
(including multinomial naive bayes, naive bayes, linear SVC, and logistic 
regression), or the classifier had peak performance near 16 features 
(including the decision tree, and random forest classifiers).  The naive bayes
classifier is a bit more difficult to place.  It technically has a maximum at
8 features, but this is unusual for Gaussian naive bayes; normally adding
more features continues to improve the accuracy but with diminishing returns.  
It seems possible that the very high performance at 8 features is due to 
variance in the error estimate.

It makes sense that the decision tree and random forest method plateau
in terms of accuracy.  These algorithms are very susceptible to overfitting
when the depth of the tree is not controlled.  Since we were focussing an 
assessing LDA accross many algoritms, which was already computationally 
intensive, we did not attempt to detect and mitigate overfitting when building
these trees for given LDA features.

Since we used Naive Bayes both with representations based on (weighted)
word counts and LDA features, we can make a comparison accross these 
representations.  When using word count-based representations, with the \textsc{tficf} weighting, stop-word removal, and including bigrams, we achieved an
accuracy of 0.839.  Using the LDA features, we achieved an optimal result
of 0.844.  These are remarkably close, with LDA being better.  In constructing 
the word-count based representations, we looked at many combinations of 
options to achieve the optimal result.  This shows that LDA does yield a 
compact representations of highly relavent features for text classification.

\section{Testing and validation}
\paragraph*{10-fold cross validation} 
With one exception, the optimizations discussed above were done with 
10-fold cross validation to estimate the error for each combination of 
representation, classifier algorithm, and parameter settings we quoted above.

\paragraph*{Validation for kNN}
The kNN algorithm used a slight variation.  During each of the 10 folds of
cross validation, an optimization was performed on the training portion of
the data for the given fold.  Using the 9/10 training data for a given fold
a nested cross-validation sub-routine was executed to determine the optimal 
number of neighbors to use, by trying all values between 5 and 30.  
The nested cross-validation was the same as the outer routine, but used only
the training data for the given fold of the outer routine.  
The optimization of $k$ was done independantly for each fold of the outer
cross-validation routine, so no information from the test set was leaked into
the training data.  Thus, this nested cross-validation procedure still
provides an estimate of the true error.

\paragraph*{Validation for ensemble classifiers}
Above, we described the optimization of a variety of classifiers.  Taking 
the predictions of each of these ``base classifiers'' as features, we then 
constructed an ensemble classifier, with the hope of improving performance
still further.

This raised an interesting difficulty in validation: because the base-level
algorithms had been optimized using the entire training set, there could be
no guarantee that cross-validation of the ensemble classifier would give 
an unbiased estimate of the true error.
Indeed, we found that the performance of the ensemble classifier in 
cross-validation was in great excess of the score that would actually be
achieved when posting results to the leaderboard.  This demonstrates how
overfitting can arise quite subtly.  For this reason, we do not report further
on the validation of the ensemble classifier.  For the purposes of the 
contest, we used leaderboard scores to test different configurations for the
ensemble classifier.


\section{Discussion}

- visualization of performance in validation

- Naive Bayes results
	remarkable increase in performance by removing stop words and using the
	tficf representation.  Adding digrams also improved the result.

- Nearest Neighbor results
	- cosine much better than euclidean
	- performance was not great overall
	- comment on the difficulty of running with large numbers of data

- LDA
	- quite good results for Decision Trees, Naive Bayes, and SVM

- SVM + tficf results

- ensemble technique

\begin{center}
	\textit{
		We hereby state that all the work presented in this report is 
		that of the authors.
	}
\end{center}

\bibliography{bib}


\end{document}
